\documentclass[dvipdfmx,a4paper,11pt]{jsarticle}

% 数式
\usepackage{amsmath,amsfonts,amssymb,mathtools,bm,titlesec,amsthm,ulem,siunitx,braket,url,physics,otf,bussproofs,enumitem,stmaryrd}
% 画像
\usepackage[dvipdfmx]{graphicx}
\usepackage{tikz}

% コード
\usepackage{listings}
\lstset{
    language=Python,
    basicstyle=\ttfamily\footnotesize,
    keywordstyle=\color{blue}\bfseries,
    commentstyle=\color{green!50!black},
    stringstyle=\color{red},
    numberstyle=\tiny\color{gray},
    numbers=left,
    numbersep=5pt,
    frame=single,
    breaklines=true,
    breakatwhitespace=true,
    tabsize=4,
    showstringspaces=false,
    backgroundcolor=\color{gray!10}
}

\mathtoolsset{showonlyrefs} % 参照した式のみ番号を表示
\everymath{\displaystyle}


\begin{document}

\title{NA課題1}
\author{05-251008 金田燈和}
\date{\today}
\maketitle

$f(x)=x^3-3x+3$とする.$f(x)=0$の根を計算する.
初期値$x_1$を$-3$から$3$の範囲で動かして,
以下のアルゴリズムを使う.アルゴリズムの収束は,
更新の量が$10^{-8}$以下かつ$f(x)$の絶対値が
$10^{-8}$以下になったことで判断する.
ただし,$x$の更新回数は$1000$を上限とする.

\begin{enumerate}
  \item 単純なニュートン法について,$x_1$の関数として,
  収束までに$x$を更新する関数をグラフに表せ.\\
  図\ref{fig:newton}に示した.
  \begin{figure}[htbp]
    \centering
    \includegraphics[width=12cm]{newton_methods_convergence.png}
    \caption{単純なニュートン法}
    \label{fig:newton}
  \end{figure}
  \item $\beta=0.2,\lambda=2.0$とした減速ニュートン法に
  ついて,$x_1$の関数として,収束までに$x$を更新する回数を
  グラフに表せ.\\
  図\ref{fig:dumped_newton}に示した.
  \begin{figure}[htbp]
    \centering
    \includegraphics[width=12cm]{dumped_newton_methods_convergence.png}
    \caption{減速ニュートン法}
    \label{fig:dumped_newton}
  \end{figure}
  \item 平野の変形ニュートン法について,$x_1$
  の関数として,収束までの$x$を更新する回数をグラフに表せ.
  パラメータは任意に選択してよいが,$x_1$に対しては定数とせよ.\\
  図\ref{fig:hirano_newton}に示した.
  なおパラメータについては,$\beta=0.2,\delta=1.0$とした.
  \begin{figure}[htbp]
    \centering
    \includegraphics[width=12cm]{hirano_newton_methods_convergence.png}
    \caption{平野の変形ニュートン法}
    \label{fig:hirano_newton}
  \end{figure}
\end{enumerate}

以上の3つのグラフを重ねたものを図\ref{fig:all_newton}に示す.
\begin{figure}[htbp]
  \centering
  \includegraphics[width=12cm]{all_newton_methods_convergence.png}
  \caption{3つのニュートン法の比較}
  \label{fig:all_newton}
\end{figure}

\newpage
\section*{実装コード}
\begin{lstlisting}
import numpy as np

beta = 0.2
lambda_ = 2
delta = 1.0

def f(x):
    return x**3 - 3*x + 3

def f_prime(x):
    ans = 3*x**2 - 3
    if ans == 0:
        return f_prime(x + 1e-6)
    return ans

def f_primes(x,n):
    if n<=0:
        return f(x)
    if n==1:
        return f_prime(x)
    if n==2:
        ans=6*x
    if n==3:
        ans=6
    if n>3:
        ans=0
    if ans == 0:
        return f_primes(x + 1e-6,n)
    return ans

def newton_step(x):
    return  x - f(x) / f_prime(x)

def dumped_newton_step(x):
    i = 0
    e = np.abs(f(x))
    while True:
        x_new = x - lambda_**(-i) * f(x) / f_prime(x)
        if np.abs(f(x_new)) <= (1 - (1 - beta) * lambda_**(-i)) * e:
            return x_new
        i += 1

def factorial(n):
    if n == 0:
        return 1
    else:
        return n * factorial(n - 1)

def hirano_newton_step(x):
    a=[f_primes(x,i) / factorial(i) for i in range(4)]
    mu=1
    e=np.abs(a[0])
    while True:
        xi=[np.pow(-mu*a[0]/a[i]+0j,1/i) for i in range(1,4)]
        min_xi=min(xi,key=lambda x:np.abs(x))
        d=sum([a[i]*np.pow(min_xi,i) for i in range(4)])
        if np.abs(d)<= (1 - (1 - beta) * mu) * e:
            return x + min_xi
        mu/=delta+1

def newton_method(step, x0, max_iter=1000, tol=1e-8):
    trajectory = [x0]
    x = x0

    for _ in range(max_iter):
        x_new = step(x)
        trajectory.append(x_new)

        if np.abs(f(x_new)) <= tol and np.abs(x_new - x) <= tol:
            break
        x = x_new
    
    return len(trajectory)

import matplotlib.pyplot as plt

x1_values = np.linspace(-3, 3, 1000)
newton_iters = [newton_method(newton_step,x1) for x1 in x1_values]
dumped_newton_iters = [newton_method(dumped_newton_step,x1) for x1 in x1_values]
hirano_newton_step_iters = [newton_method(hirano_newton_step,x1) for x1 in x1_values]

plt.plot(x1_values, newton_iters, label='Newton Method')
plt.plot(x1_values, dumped_newton_iters, label='Dumped Newton Method')
plt.plot(x1_values, hirano_newton_step_iters, label='Hirano Modified Newton Method')
plt.xlabel('Initial Value x1')
plt.ylabel('Number of Iterations to Converge')
plt.yscale('log')
plt.legend()
plt.title('Convergence of Newton Methods')
plt.grid()
# plt.show()
plt.savefig('all_newton_methods_convergence.png')
\end{lstlisting}


\end{document}